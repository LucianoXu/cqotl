
Extending Quantum Relational Hoare Logics from Optimal Transport Duality with Classic variables and QProver implementation that generates proof obligations resolved by the formalization in the LEAN proof assistant.

% Contributions
% \begin{enumerate}
%     \item A logic with qWhile with classic variables
%     \item Implementation in OCaml to generate proof obligations
%     \item Formalized assertion and resolution of proof obligations in LEAN4
%     \item Examples: 3 to 4
% \end{enumerate}


\section{Type System}
This section introduces the typing system of the prover, which includes the types, terms and propositions. We use $x$ to denote the variables.

\newcommand{\Type}{\textsf{Type}}
\newcommand{\CType}{\textsf{CType}}
\newcommand{\CTerm}[1]{\ensuremath{\textsf{CTerm}(#1)}}
\newcommand{\SType}{\ensuremath{\mathcal{S}}}

\newcommand{\cvar}[1]{\ensuremath{\textsf{CVar}(#1)}}
\newcommand{\qreg}[1]{\ensuremath{\textsf{QReg}(#1)}}
\newcommand{\qvlist}{\textsf{QVList}}
\newcommand{\opt}[2]{\ensuremath{\mathcal{O}(#1, #2)}}
\newcommand{\ldirac}[2]{\ensuremath{\mathcal{D}(#1, #2)}}

\newcommand{\optpair}{\ensuremath{\textsf{OptPair}}}
\newcommand{\unitary}[1]{\ensuremath{\textsf{Unitary}({#1})}}
\newcommand{\assn}[1]{\ensuremath{\textsf{Assn}(#1)}}
\newcommand{\meas}[1]{\ensuremath{\textsf{Meas}(#1)}}
\newcommand{\prog}{\ensuremath{\textsf{Prog}}}
\newcommand{\prop}{\ensuremath{\textsf{Prop}}}
\newcommand{\bit}{\textsf{Bit}}
\newcommand{\pf}{\textsf{proof}}
\newcommand{\judge}[4]{\ensuremath{\{#1\}~#2 \sim #3~\{#4\}}}

\begin{definition}
    \begin{align*}
        t ::=\ & x \\
        & \mid \Type \mid \prop \mid \qvlist \mid \optpair \\
        & \mid \CType \mid \cvar{t} \mid \qreg{t} \mid \prog \\
        & \mid \bit \\
        & \mid \CTerm{t} \mid \SType \mid \opt{t}{t} \mid \ldirac{t}{t} \\
        & \mid t \times t  \\ 
        & \mid (t, t) \mid [x_1, x_2, \dots, x_n] \mid t_{t,t} \\
        & \mid opt \mid stmts \mid \phi \\
        & \mid \pf
    \end{align*}
\end{definition}

\subsection{Operators}

\begin{definition}
    \begin{align*}
        opt ::= & \ t + t
    \end{align*}
\end{definition}

\subsection{Program statements}
qwhile programs are encoded as terms in the prover.

\newcommand{\rndarrow}{%
  \stackrel{\raisebox{-.25ex}[.25ex]%
   {\tiny $\mathdollar$}}{\raisebox{-.25ex}[.25ex]{$\leftarrow$}}}

\newcommand{\Skip}{\textbf{skip}}
\newcommand{\Assign}[2]{\ensuremath{#1:=#2}}
\newcommand{\pAssign}[2]{\ensuremath{#1 \rndarrow #2}}
\newcommand{\Init}[1]{\ensuremath{#1:=\ket{0}}}
\newcommand{\MeaAssign}[2]{\ensuremath{#1 := \textbf{meas}\ #2}}
\newcommand{\If}[3]{\ensuremath{
    \textbf{if}\ #1\ \textbf{then}\ #2\ \textbf{else}\ #3\ \textbf{end}
}}
\newcommand{\While}[2]{\ensuremath{
    \textbf{while}\ #1\ \textbf{do}\ #2\ \textbf{end}
}}

\begin{definition}[program syntax]
    \label{def: prog syntax}
    The program statements are defined as follows:
    \begin{align*}
        & stmts ::= && stmt \mid stmts\ stmt & \text{(statement sequence)} \\
        & stmt ::= && \Skip & \text{(skip)} \\
            & && \mid \Assign{x}{t} & \text{(deterministic assignment)} \\
            & && \mid \pAssign{x}{T} & \text{(probabilistic assignment)} \\
            & && \mid \Init{t} & \text{(initialization)} \\
            & && \mid U\ t & \text{(unitary transformation)} \\
            & && \mid \MeaAssign{x}{t} & \text{(quantum measurement)} \\
            & && \mid \If{t}{stmts_1}{stmts_2} & \text{(if statement)} \\
            & && \mid \While{t}{stmts} & \text{(while statement)}
    \end{align*}
\end{definition}

\subsection{Propositions}

\begin{definition}[proposition syntax]
    \label{def:prop syntax}
    \begin{align*}
        \phi ::=\ & \unitary{t} \mid \assn{t} \mid \meas{t} \mid \judge{P}{s_1}{s_2}{Q} \\
                & \mid t = t
    \end{align*}
\end{definition}
Here $P$ and $Q$ are operators. $s_1$ and $s_2$ are program terms.

\textbf{Remark:} Here the arguments in constructors like $\judge{P}{s_1}{s_2}{Q}$ are hinted to be particular types, but this is only for elaboration consideration. We adopt the general approach in the implementation, which allows the arguments to be any term in syntax, and validity is guaranteed by type checking.


\subsection{Typing}

\newcommand{\WF}[1]{\ensuremath{\mathcal{WF}(#1)}}

We maintain a context $\Gamma$ for the typing of terms and the proof of propositions. 
A context $\Gamma$ consists of a list of declarations, which are definitions $x := t : T$ or assumptions $x : T$. 
We use $[]$ to denote the empty context, and $x \in \Gamma$ to denote that variable $x$ appears in the context as a declaration.
We only work with well-formed contexts, denoted as \WF{\Gamma}. 
The typing judgement, denoted as $\Gamma \vdash t : T$, represents that the term $t$ has type $T$ in well-formed context $\Gamma$. 


The well-formed contexts can be constructed in the following ways:

\begin{gather*}
    \frac{}{\WF{[]}} \\
    \\
    \frac{\WF{\Gamma} \quad x \notin \Gamma \quad \Gamma \vdash T : \Type \text{ or } \Gamma \vdash T : \prop}{\WF{\Gamma;x:T}} \\
    \\
    \frac{\WF{\Gamma} \qquad x \notin \Gamma \qquad \Gamma \vdash t : T}{\WF{\Gamma;x := t : T}} \\
\end{gather*}

These are subtyping rules:
\begin{gather*}
    \frac{\Gamma \vdash t : \cvar{t}}{\Gamma \vdash t : \CTerm{t}}
\end{gather*}


The typing rules for type terms are as follows.
\begin{gather*}
    \frac{\WF{\Gamma}}{\Gamma \vdash \prop : \Type}
    \qquad
    \frac{\WF{\Gamma}}{\Gamma \vdash \qvlist : \Type}
    \qquad
    \frac{\WF{\Gamma}}{\Gamma \vdash \optpair : \Type}
    \qquad
    \frac{\WF{\Gamma}}{\Gamma \vdash \CType : \Type} \\
    \\
    \frac{\Gamma \vdash T : \CType}{\Gamma \vdash \cvar{T} : \Type}
    \qquad
    \frac{\Gamma \vdash T : \CType}{\Gamma \vdash \qreg{T} : \Type}
    \qquad
    \frac{\WF{\Gamma}}{\Gamma \vdash \prog : \Type} \\
    \\
    \frac{\WF{\Gamma}}{\Gamma \vdash \bit : \CType}
    \qquad
    \frac{\Gamma \vdash T : \CType}{\Gamma \vdash \CTerm{T} : \Type }
    \qquad
    \frac{\WF{\Gamma}}{\Gamma \vdash \SType : \Type} \\
    \\
    \frac{\Gamma \vdash T_1 : \CType \quad \Gamma \vdash T_2 : \CType}{\Gamma \vdash \opt{T_1}{T_2} : \Type}
    \qquad
    \frac{\Gamma \vdash r_1 : \qvlist \quad \Gamma \vdash r_2 : \qvlist}{\Gamma \vdash \ldirac{r_1}{r_2} : \Type} \\
    \\
    \frac{\Gamma \vdash T_1 : \CType \quad \Gamma \vdash T_2 : \CType}{\Gamma \vdash T_1 \times T_2 : \CType} \\
    \\
    \frac{
    \Gamma \vdash t_1 : \CTerm{T_1} \quad \Gamma \vdash t_2 : \CTerm{T_2}
    }
    {\Gamma \vdash (t_1, t_2) : \CTerm{T_1 \times T_2}}
    \qquad
    \frac{\Gamma \vdash o_1 : \ldirac{r_1}{r_2} \quad \Gamma \vdash o_2 : \ldirac{r_1'}{r_2'}}{\Gamma \vdash (o_1, o_2) : \optpair} \\
    \\
    \frac{\Gamma \vdash r_1 : \qreg{T_1} \quad \Gamma \vdash r_2 : \qreg{T_2}}{\Gamma \vdash (r_1, r_2) : \qreg{T_1 \times T_2}}
    \\
    \\
    \frac{\Gamma \vdash x_i : \qreg{\sigma_i} \text{ for all } i \qquad x_i \neq x_j \text{ for all } i \neq j}{\Gamma \vdash [x_1, x_2, \dots, x_n] : \qvlist} \\
    \\
    \frac{\Gamma \vdash o : \opt{T_1}{T_2} \quad \Gamma \vdash r_1 : \qreg{T_1} \quad \Gamma \vdash r_2 : \qreg{T_2}}{\Gamma \vdash o_{r_1,r_2} : \ldirac{\textsf{qvars}\ r_1}{\textsf{qvars}\ r_2}}
\end{gather*}

The typing rules for Dirac notations are
\begin{gather*}
    \frac{\Gamma \vdash o_1 : \opt{T_1}{T_2} \quad \Gamma \vdash o_2 : \opt{T_1}{T_2}}{\Gamma \vdash o_1 + o_2 : \opt{T_1}{T_2}}
\end{gather*}


The typing rules for program statements are:
\begin{gather*}
    \frac{\WF{\Gamma}}{\Gamma \vdash \Skip : \prog} \\
    \\
    \frac{\Gamma \vdash x : \cvar{T} \quad \Gamma \vdash t : \CTerm{T}}{\Gamma \vdash \Assign{x}{t} : \prog} \\
    \\
    \frac{\Gamma \vdash x : \cvar{T}}{\Gamma \vdash \pAssign{x}{T} : \prog} \\
    \\
    \frac{\Gamma \vdash t : \qreg{T}}{\Gamma \vdash \Init{t} : \prog} \\
    \\
    \frac{\Gamma \vdash U : \opt{T}{T} \quad \Gamma \vdash t : \qreg{T} \quad \Gamma \vdash \unitary{U}}{\Gamma \vdash U\ t : \prog} \\
    \\
    \frac{\Gamma \vdash x : \cvar{\bit} \quad \Gamma \vdash M : \optpair \quad \Gamma \vdash \meas{M}}{\Gamma \vdash \MeaAssign{x}{M} : \prog} \\
    \\
    \frac{\Gamma \vdash t : \CTerm{\bit} \quad \Gamma \vdash s_1 : \prog \quad \Gamma \vdash s_2 : \prog}
    {\Gamma \vdash \If{t}{s_1}{s_2} : \prog} \\
    \\
    \frac{\Gamma \vdash t : \CTerm{\bit} \quad \Gamma \vdash s : \prog}
    {\Gamma \vdash \While{t}{s} : \prog}
\end{gather*}



The typing rules for propositions are
\begin{gather*}
    \frac{\Gamma \vdash o : \opt{\sigma}{\sigma}}{\Gamma \vdash \unitary{o} : \prop}
    \qquad
    \frac{\Gamma \vdash lo : \ldirac{r_1}{r_2}}{\Gamma \vdash \assn{lo} : \prop} \\
    \\
    \frac{\Gamma \vdash M : \optpair}{\Gamma \vdash \meas{M} : \prop}
    \qquad
    \frac{
        \begin{aligned}
            & \Gamma \vdash \assn{P} \qquad \Gamma \vdash \assn{Q} \\
            & \Gamma \vdash s_1 : \prog \qquad \Gamma \vdash s_2 : \prog
        \end{aligned}}
        {
            \Gamma \vdash \judge{P}{s_1}{s_2}{Q} : \prop
        } \\
    \\
    \frac{\Gamma \vdash o_1 : \opt{T_1}{T_2} \qquad \Gamma \vdash o_2 : \opt{T_1}{T_2}}{\Gamma \vdash o_1 = o_2 : \prop}
\end{gather*}


The propositions can be proved using the following rules:
\begin{gather*}
    ...
\end{gather*}