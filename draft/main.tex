
\documentclass[acmsmall]{acmart}

%%%%%%%%%%%%%%%%%%%%%%%%%%%%%%%%%%


\usepackage{color}
\usepackage{amsmath}
%\usepackage{amssymb}
\usepackage{graphicx}
\usepackage{amsthm}
\usepackage{stmaryrd}
\usepackage[all]{xy}
\usepackage{multirow}
\usepackage{paralist}
\usepackage{hhline}
\usepackage{bm}
\usepackage{braket}
\usepackage{pifont}% http://ctan.org/pkg/pifont
\newcommand{\cmark}{\CIRCLE}%
\newcommand{\xmark}{\Circle}%
\newcommand{\hmark}{\LEFTcircle}%
\renewcommand{\matrix}[1]{\begin{bmatrix}#1\end{bmatrix}}
\usepackage{wasysym}
\usepackage{extarrows}
\usepackage{tikz}
\usetikzlibrary{positioning, shapes.geometric, graphs}
\usepackage{wrapfig}
\usepackage{dsfont}
\usepackage{mdframed}
\usepackage{tabularx}
\usepackage{multicol}
\usepackage{tikz}
\usetikzlibrary{decorations.pathmorphing}
\usepackage{tikz-3dplot}
\usetikzlibrary{arrows.meta}
\usetikzlibrary{shadings,shapes,arrows,decorations.pathmorphing,backgrounds,positioning,fit,petri,calc}
\usepackage{setspace}
\usepackage{cleveref}

% for Mathematica code
\usepackage{xcolor}
\usepackage{listings}

% quantum circuit graph
\usepackage{quantikz}

\usepackage{extarrows}
\usepackage{cleveref}
\usepackage[absolute,overlay]{textpos}

\usepackage{xspace}
\usepackage{graphicx}
\usepackage{subcaption}


%%
%% \BibTeX command to typeset BibTeX logo in the docs
\AtBeginDocument{%
  \providecommand\BibTeX{{%
    Bib\TeX}}}

%% Rights management information.  This information is sent to you
%% when you complete the rights form.  These commands have SAMPLE
%% values in them; it is your responsibility as an author to replace
%% the commands and values with those provided to you when you
%% complete the rights form.

\setcopyright{rightsretained}
\acmJournal{PACMPL}
\acmYear{2025} \acmVolume{9} \acmNumber{POPL} \acmArticle{42} \acmMonth{1}\acmDOI{10.1145/3704878}



\newmdenv[
  skipabove=5pt,
  skipbelow=5pt,
  leftmargin=0pt,
  rightmargin=0pt,
  innerleftmargin=5pt,
  innerrightmargin=5pt,
  innertopmargin=5pt,
  innerbottommargin=5pt,
  linecolor=black,
]{framedeq}



\newcommand{\yx}[1]{\textit{\color{blue}[YX] : #1}}
\newcommand{\gb}[1]{\textit{\color{red}[GB] : #1}}
\newcommand{\lz}[1]{\textit{\color{brown}[LZ] : #1}}


\newcommand*{\Sc}{\mathcal{S}}
\newcommand*{\K}{\mathcal{K}}
\newcommand*{\B}{\mathcal{B}}
\newcommand*{\Op}{\mathcal{O}}

\newcommand*{\type}{\texttt{type}}

\newcommand*{\unit}{\texttt{unit}}
\newcommand*{\utt}{\texttt{tt}}
\newcommand*{\fst}{\texttt{fst }}
\newcommand*{\snd}{\texttt{snd }}
\newcommand*{\reduce}{\ \triangleright\ }
\newcommand*{\reducefrom}{\ \triangleleft\ }

\newcommand*{\zeroK}[1]{\mathbf{0}_{\mathcal{K}(#1)}}
\newcommand*{\zeroB}[1]{\mathbf{0}_{\mathcal{B}(#1)}}
\newcommand*{\zeroO}[1]{\mathbf{0}_{\mathcal{O}(#1)}}

\newcommand*{\sem}[1]{{\llbracket #1 \rrbracket}}

\newcommand*{\done}{\textcolor{blue}{\textbf{ [DONE] }}}
\newcommand*{\doing}{\textcolor{red}{\textbf{ [DOING] }}}
\newcommand*{\pending}{\textcolor{gray}{\textbf{ [PENDING] }}}

\newcommand*{\DN}{\textsf{DN}}
\newcommand*{\DNE}{\textsf{DNE}}
\newcommand*{\CIME}{\texttt{CiME2}}
\newcommand*{\APROVE}{\texttt{AProVE}}
\newcommand*{\DIRACDEC}{\texttt{DiracDec}}
\newcommand*{\CC}{\mathbb{C}}

\title{Complete Classical-Quantum Relational Hoare Logics from Optimal Transport Duality}

\begin{abstract}
Extending Quantum Relational Hoare Logics from Optimal Transport Duality with Classic variables and QProver implementation that generates proof obligations resolved by the formalization in the LEAN proof assistant.
\end{abstract}

%%
%% The code below is generated by the tool at http://dl.acm.org/ccs.cfm.
%% Please copy and paste the code instead of the example below.
%%

% \begin{CCSXML}
%     <ccs2012>
%       <concept>
%         <concept_id>10010147.10010148.10010162.10010163</concept_id>
%         <concept_desc>Computing methodologies~Special-purpose algebraic systems</concept_desc>
%         <concept_significance>500</concept_significance>
%       </concept>
%       <concept>
%         <concept_id>10003752.10003766.10003767.10003769</concept_id>
%         <concept_desc>Theory of computation~Rewrite systems</concept_desc>
%         <concept_significance>500</concept_significance>
%       </concept>
%       <concept>
%         <concept_id>10003752.10003753.10003758</concept_id>
%         <concept_desc>Theory of computation~Quantum computation theory</concept_desc>
%         <concept_significance>300</concept_significance>
%       </concept>
%     </ccs2012>
% \end{CCSXML}
  
%   \ccsdesc[500]{Computing methodologies~Special-purpose algebraic systems}
%   \ccsdesc[500]{Theory of computation~Rewrite systems}
%   \ccsdesc[300]{Theory of computation~Quantum computation theory}

\begin{document}

\maketitle


\section{Introduction}
Contributions
\begin{enumerate}
    \item A logic with qWhile with classic variables
    \item Implementation in OCaml to generate proof obligations
    \item Formalized assertion and resolution of proof obligations in LEAN4
    \item Examples: 3 to 4
\end{enumerate}

\section{Background}


\section{Motivating Examples}

\subsection{Example 1}

\subsection{Example 2}

\subsection{Example 3}


\section{Prover: Syntax, Typing and Verification Conditions Generation}
This section introduces the basic language syntax of the Prover, its typing rules and the verification condition generation mechanism. 

\subsection{Syntax}
\begin{definition}[Syntax] 
\label{def:syntax}
The QProver syntax is defined inductively as follows:
\begin{align*}
    & && \textrm{(Types)}               && A &&::= qvar ~|~ qreg(n) ~|~ \Op(n) ~|~ \Op_l(n) ~|~ \tau ~|~ q ~|~ \{\tau_1\} ~q_1 \sim q_2~\{\tau_2\} ~|~ P \\
    & && \textrm{(Meta Expressions)}    && O &&::= Var~x~:A ~|~ Def ~name~:=~E ~|~Prove~ name~(\{P\} ~E_1 \sim E_2~\{Q\}). ~Proof \\& && && && ~|~ O_1;O_2\\
    & && \textrm{(Expressions)}     && E &&::= \textbf{\texttt{skip}} ~|~ \texttt{$E_1$;$E_2$} ~|~ q~:=~\ket{0} ~|~ \bar{q} :=U[\bar{q}] ~|~ \texttt{if} ~(\square m . \textit{M}[\bar{q}] = m\rightarrow P_m) ~\texttt{fi} \\& && && &&
    ~|~ \texttt{while} ~\textit{M}[\bar{q}] = 1 ~do~ P ~od \\
    & && \textrm{(Assertion)}           && P,Q && To ~be~defined \\
    & && \textrm{(Context)} && \Gamma && ::= \emptyset~|~ \Gamma , x:A
\end{align*}
\end{definition}

\paragraph{Types.}
\paragraph{Meta Expressions.}
\paragraph{Expressions.}
\paragraph{Assertion.}

\subsection{Typing}
\begin{definition}[Well-typed meta expressions]
    A meta expression $O$ is valid if it has a type $A$ described by the typing judgment $\Gamma \vdash O:A$, where $\Gamma$ is the typing context.
\end{definition}

\begin{figure}[t]
 \centering
 \small
    %\begin{spacing}{2}
 \begin{align*}
        & \textsc{(Context)} &&
 \frac{x : A \in \Gamma}{\Gamma \vdash x : A}
 \qquad \qquad \qquad \qquad 
 \\[0.1cm]
&\textsc{(Meta ~Expressions)}&& 
    \frac{x:A \in \Gamma}{\Gamma \vdash Var ~x:A}
    \quad \quad
    \frac{\Gamma \vdash E : A}{\Gamma \vdash Def ~name~:=E~:~A} 
    \\[0.1cm]
    & && \frac{\Gamma \vdash \{P\} ~E_1 \sim ~E_2~\{Q\}: \{\tau_1\} ~q_1 \sim ~q_2~\{\tau_2\}}{\Gamma \vdash Prove ~name~ : (\{P\} ~E_1 \sim ~E_2~\{Q\}):P} \\[0.1cm]
    & && \frac{\Gamma\vdash O_1:A \quad \Gamma,O_1:A \vdash O_2 : B}{\Gamma \vdash O_1; O_2}
 \end{align*}
 \caption{Typing rules for the QProver.}
    \label{fig: DN typing}
\end{figure}


\section{Prover}


\section{Theory}


\section{Dirac Notation Extended Language}

\section{Implementation and Mechanization}


\section{Evaluation}


\section{Related Work}


\section*{Acknowledgement}

\newpage 
\bibliographystyle{ACM-Reference-Format}
\bibliography{ref}

\clearpage
\appendix

\appendix

\section{To be added}

\end{document}
\endinput
